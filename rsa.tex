% This is LLNCS.DEM the demonstration file of
% the LaTeX macro package from Springer-Verlag
% for Lecture Notes in Computer Science,
% version 2.4 for LaTeX2e as of 16. April 2010
%
\documentclass{llncs}
%
\usepackage{makeidx}  % allows for indexgeneration
\usepackage{amsmath}
%
\begin{document}
\mainmatter% start of the contributions
%
\title{Review --- A Method for Obtaining Digital Signatures and Public-Key Cryptosystems}
%
\titlerunning{Graham Scan}  % abbreviated title (for running head)
%                                     also used for the TOC unless
%                                     \toctitle is used
%
\author{Daniella Albuquerque dos Angelos}
%
\authorrunning{Daniella Albuquerque dos Angelos} % abbreviated author list (for running head)
%
\institute{Universidade de Brasilia\\
\email{daniaangelos@gmail.com} }

\maketitle              % typeset the title of the contribution

\begin{abstract}
\keywords{computational geometry, graph theory, Hamilton cycles}
\end{abstract}
%
\section{Introduction}
\label{intro}
%Motivation

% Public-Key Cryptosystems

%Organization

\section{RSA Encryption and Decryption Methods}
\label{rsa}

To encrypt a message $M$, using a public encryption key $(e, n)$, being $e$ and
$n$ positive integers, proceed as follows.

First, use any standard representation to represent the message as an integer
between 0 and $n - 1$. The purpose here is not to encrypt the message but only
to get it into the numeric form necessary for encryption.

Then, encrypt the message by raising it to the $e$-th power modulo $n$. That is,
the ciphertext $C$ is the remainder when $M^e$ is divided by $n$.

To decrypt the ciphertext, raise it to another power $d$, again modulo $n$. The
encryption and decryption algorithms $E$ and $D$ are thus:

\begin{align*}
    C \equiv  E(M) & \equiv M^e\ \text{(mod $n$), for a message $M$}\\
     D(C) & \equiv  C^d\ \text{(mod $n$), for a ciphertext $C$}
\end{align*}

Note that encryption does not increase the size of a message.

The \emph{encryption key} is thus the pair $(e, n)$. Similarly, the
\emph{decryption key} is the pair $(d, n)$. Each user makes his encryption key
public, and keeps the corresponding decryption key private.

To choose the appropriate keys to use this method, one first needs to compute
$n$ as the product of two large random primes $p$ and $q$:

\begin{equation*}
   n = p \cdot q 
\end{equation*}

Although $n$ will be made public, the prime factors $p$ and $q$ are both
hidden due to the enormous difficulty of factoring $n$, that we are aware of.
This also hides the way $d$ can be derived from $e$. Then, a choice for $d$ is
any random large integer which is relatively prime to $(p - 1)\cdot(q - 1)$.
That is, $d$ satisfies:
\begin{equation*}
    \gcd(d, (p-1)(q-1)) = 1   
\end{equation*}
where $\gcd$ means the greatest common divisor.

\subsection{Algorithms}
\label{algorithms}

\section{Complexity}
\label{complexity}

\section{Related Work}
\label{work}


%
% ---- Bibliography ----
%
\begin{thebibliography}{5}
%
\bibitem{clar:eke}
Clarke, F., Ekeland, I.:
Nonlinear oscillations and
boundary-value problems for Hamiltonian systems.
Arch. Rat. Mech. Anal. 78, 315--333 (1982)

\bibitem{clar:eke:2}
Clarke, F., Ekeland, I.:
Solutions p\'{e}riodiques, du
p\'{e}riode donn\'{e}e, des \'{e}quations hamiltoniennes.
Note CRAS Paris 287, 1013--1015 (1978)

\bibitem{mich:tar}
Michalek, R., Tarantello, G.:
Subharmonic solutions with prescribed minimal
period for nonautonomous Hamiltonian systems.
J. Diff. Eq. 72, 28--55 (1988)

\bibitem{tar}
Tarantello, G.:
Subharmonic solutions for Hamiltonian
systems via a $\bbbz_{p}$ pseudoindex theory.
Annali di Matematica Pura (to appear)

\bibitem{rab}
Rabinowitz, P.:
On subharmonic solutions of a Hamiltonian system.
Comm. Pure Appl. Math. 33, 609--633 (1980)

\end{thebibliography}


\end{document}
